%# -*- coding: utf-8-unix -*-
%%==================================================
%% abstract.tex for SJTU Master Thesis
%%==================================================

\begin{abstract}
\thispagestyle{front}

最近几年来,以Docker为代表的容器虚拟化技术越来越被业界所接受,成为部署应用时的另一种选择。而容器集群技术使得开发者无需关注底层架构与环境的情况下,快速部署自己的应用。

容器和容器集群使得应用的部署变得简单,而本文希望能够在基于容器集群的架构上,实现从代码提交,到持续集成,再到最后的持续部署发布的全流程的自动化系统Fornax,进一步简化运维工作。Fornax在构建阶段使用了容器虚拟化技术来进行构建的隔离,并且在每次构建后产出一个版本镜像,实现了代码与运行环境两者的共同管理。而持续集成与持续部署,是建立在Kubernetes容器集群上的。

在本文的最后部分,对Fornax的功能性测试以及分布式的部署进行了探索,保证了Fornax的功能的合约以及在生产环境下的可用性。目前Fornax已经在线上运行,为用户提供服务,这也从另一方面证明了Fornax的可用性。

\keywords{\large 容器虚拟化 \quad 版本管理 \quad 持续集成}
\end{abstract}

\begin{englishabstract}
\thispagestyle{front}



\englishkeywords{\large Containerization, Version Control, Continuous Integration}
\end{englishabstract}

