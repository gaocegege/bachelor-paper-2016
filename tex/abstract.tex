%# -*- coding: utf-8-unix -*-
%%==================================================
%% abstract.tex for SJTU Master Thesis
%%==================================================

\begin{abstract}
\thispagestyle{front}

最近几年来,以Docker为代表的容器虚拟化技术越来越被业界所接受,成为部署应用时的另一种选择。而容器集群技术使得开发者无需关注底层架构与环境的情况下,快速部署自己的应用,快速响应需求。

容器和容器集群使得应用的部署变得简单,而本文希望能够在基于容器集群的架构上,实现从代码提交,到持续集成,再到最后的持续部署发布的全流程的自动化系统Fornax,进一步简化运维工作。Fornax在构建阶段使用了容器虚拟化技术来进行构建的隔离,并且在每次构建后产出一个版本镜像,实现了代码与运行环境两者的共同管理。而持续集成与持续部署,是建立在Kubernetes容器集群上的。

在本文的最后部分,对Fornax的功能性测试以及分布式的部署进行了探索,保证了Fornax的功能的合约以及在生产环境下的可用性。目前Fornax已经在线上运行,为用户提供服务,这也从另一方面证明了Fornax的可用性。

\keywords{\large 容器虚拟化 \quad 版本管理 \quad 持续集成}
\end{abstract}

\begin{englishabstract}
\thispagestyle{front}

In recent years, operating-system-level virtualization is increasingly being accepted by the industry, and has become another option when deploying applications. The clustering system helps programmers to respond to customer demand quickly and efficiently.

Operating-system-level virtualization makes applications easier to deploy, and this paper want to implement a automation system for continuous integration, continuous deployment and version release management. Fornax uses operating-system-level virtualization to isolate the building tasks. And When the building task is done, Fornax would push a image to registry, makes both the code and runtime under control. And continuous integration and continuous deployment are based on Kubernetes.

In the end, Fornax introduces the end-to-end test and distributed architecture, to illustrate the functionality and usability. Now Fornax is running in production, on the other hand it shows the availability of Fornax.

\englishkeywords{\large Containerization, Version Control, Continuous Integration}
\end{englishabstract}
