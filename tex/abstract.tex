%# -*- coding: utf-8-unix -*-
%%==================================================
%% abstract.tex for SJTU Master Thesis
%%==================================================

\begin{abstract}

最近几年来,以Docker为代表的容器虚拟化技术越来越被业界所接受,成为部署应用时的另一种选择。以容器的方式进行部署,具有跨平台、较低的资源损耗、较好的隔离性的优点。

本文分析了传统的软件过程,以及容器虚拟化技术对当下软件工程的影响,并基于容器虚拟化技术和容器集群,实现了从代码提交,到持续集成,再到最后的持续部署发布的版本管理与发布系统Fornax。Fornax具有三个特性,持续集成,持续部署与版本管理。传统的持续集成与持续部署工具,只关注于持续集成与持续部署,而Fornax不仅是一个持续集成与持续部署的工具,也关注于对于版本的管理。

Fornax在构建阶段使用了容器虚拟化技术来进行构建的隔离,并且在每次构建后产出一个版本镜像,实现了代码与运行环境两者的共同管理。并在最后,进行了针对Fornax的功能性测试以及分布式部署的探索,保证了Fornax的功能的合约以及在生产环境下的可用性。

\keywords{\large 容器虚拟化 \quad 版本管理 \quad 持续集成}
\end{abstract}

\begin{englishabstract}

In recent years, operating-system-level virtualization is increasingly being accepted by the industry, and has become another option when deploying applications. It has the advantage of cross-platform, low resource consumption, and better isolation.

This paper analyzes the traditional software process, and the operating-system-level virtualization. Then we build a version release system called Fornax. Fornax has three features: continuous integration, continuous deployment and version management.Traditional continuous integration and continuous deployment tools only care about continuous integration or continuous deployment, and Fornax focus on the version management.

Fornax uses operating-system-level virtualization to isolate the building tasks. And When the building task is done, Fornax would push a image to registry, makes both the code and runtime under control. At the end, This paper introduces the end-to-end test and distributed architecture, to illustrate the functionality and usability.

\englishkeywords{\large Containerization, Version Control, Continuous Integration}
\end{englishabstract}

