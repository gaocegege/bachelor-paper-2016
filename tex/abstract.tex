%# -*- coding: utf-8-unix -*-
%%==================================================
%% abstract.tex for SJTU Master Thesis
%%==================================================

\begin{abstract}

最近几年来,以Docker为代表的容器虚拟化技术越来越被业界所接受,成为部署应用时的另一种选择。以容器的方式进行部署,具有跨平台、较低的资源损耗、较好的隔离性的优点。但是,因为容器虚拟化技术相对于传统的虚拟化技术而言,并没有经过大规模的生产环境的使用测试,因此没有形成从代码提交到应用部署的一整套软件开发过程,而本课题就专注于如何借助容器虚拟化的技术,构建一个基于容器集群的版本管理与发布系统。

本文分析了传统的软件过程,并基于容器虚拟化技术和容器集群,实现了从代码提交,到持续集成,再到最后的持续部署发布的版本管理与发布系统Fornax。相比于之前的系统,Fornax在构建阶段使用了容器虚拟化技术来进行构建的隔离,并且在每次构建后产出一个版本镜像,实现了代码与运行环境两者的共同管理。

并在最后,进行了针对Fornax的功能性测试以及分布式部署的探索,保证了Fornax的功能的合约以及在生产环境下的可用性。

\keywords{\large 容器虚拟化 \quad 版本管理 \quad 持续集成}
\end{abstract}

\begin{englishabstract}



\englishkeywords{\large Containerization, Version Control, Continuous Integration}
\end{englishabstract}

